% Template for PLoS
% Version 3.5 March 2018
%
% % % % % % % % % % % % % % % % % % % % % %
%
% -- IMPORTANT NOTE
%
% This template contains comments intended
% to minimize problems and delays during our production
% process. Please follow the template instructions
% whenever possible.
%
% % % % % % % % % % % % % % % % % % % % % % %
%
% Once your paper is accepted for publication,
% PLEASE REMOVE ALL TRACKED CHANGES in this file
% and leave only the final text of your manuscript.
% PLOS recommends the use of latexdiff to track changes during review, as this will help to maintain a clean tex file.
% Visit https://www.ctan.org/pkg/latexdiff?lang=en for info or contact us at latex@plos.org.
%
%
% There are no restrictions on package use within the LaTeX files except that
% no packages listed in the template may be deleted.
%
% Please do not include colors or graphics in the text.
%
% The manuscript LaTeX source should be contained within a single file (do not use \input, \externaldocument, or similar commands).
%
% % % % % % % % % % % % % % % % % % % % % % %
%
% -- FIGURES AND TABLES
%
% Please include tables/figure captions directly after the paragraph where they are first cited in the text.
%
% DO NOT INCLUDE GRAPHICS IN YOUR MANUSCRIPT
% - Figures should be uploaded separately from your manuscript file.
% - Figures generated using LaTeX should be extracted and removed from the PDF before submission.
% - Figures containing multiple panels/subfigures must be combined into one image file before submission.
% For figure citations, please use "Fig" instead of "Figure".
% See http://journals.plos.org/plosone/s/figures for PLOS figure guidelines.
%
% Tables should be cell-based and may not contain:
% - spacing/line breaks within cells to alter layout or alignment
% - do not nest tabular environments (no tabular environments within tabular environments)
% - no graphics or colored text (cell background color/shading OK)
% See http://journals.plos.org/plosone/s/tables for table guidelines.
%
% For tables that exceed the width of the text column, use the adjustwidth environment as illustrated in the example table in text below.
%
% % % % % % % % % % % % % % % % % % % % % % % %
%
% -- EQUATIONS, MATH SYMBOLS, SUBSCRIPTS, AND SUPERSCRIPTS
%
% IMPORTANT
% Below are a few tips to help format your equations and other special characters according to our specifications. For more tips to help reduce the possibility of formatting errors during conversion, please see our LaTeX guidelines at http://journals.plos.org/plosone/s/latex
%
% For inline equations, please be sure to include all portions of an equation in the math environment.
%
% Do not include text that is not math in the math environment.
%
% Please add line breaks to long display equations when possible in order to fit size of the column.
%
% For inline equations, please do not include punctuation (commas, etc) within the math environment unless this is part of the equation.
%
% When adding superscript or subscripts outside of brackets/braces, please group using {}.
%
% Do not use \cal for caligraphic font.  Instead, use \mathcal{}
%
% % % % % % % % % % % % % % % % % % % % % % % %
%
% Please contact latex@plos.org with any questions.
%
% % % % % % % % % % % % % % % % % % % % % % % %

\documentclass[10pt,letterpaper]{article}
\usepackage[top=0.85in,left=2.75in,footskip=0.75in]{geometry}

% amsmath and amssymb packages, useful for mathematical formulas and symbols
\usepackage{amsmath,amssymb}

% Use adjustwidth environment to exceed column width (see example table in text)
\usepackage{changepage}

% Use Unicode characters when possible
\usepackage[utf8x]{inputenc}

% textcomp package and marvosym package for additional characters
\usepackage{textcomp,marvosym}

% cite package, to clean up citations in the main text. Do not remove.
% \usepackage{cite}

% Use nameref to cite supporting information files (see Supporting Information section for more info)
\usepackage{nameref,hyperref}

% line numbers
\usepackage[right]{lineno}

% ligatures disabled
\usepackage{microtype}
\DisableLigatures[f]{encoding = *, family = * }

% color can be used to apply background shading to table cells only
\usepackage[table]{xcolor}

% array package and thick rules for tables
\usepackage{array}

% create "+" rule type for thick vertical lines
\newcolumntype{+}{!{\vrule width 2pt}}

% create \thickcline for thick horizontal lines of variable length
\newlength\savedwidth
\newcommand\thickcline[1]{%
  \noalign{\global\savedwidth\arrayrulewidth\global\arrayrulewidth 2pt}%
  \cline{#1}%
  \noalign{\vskip\arrayrulewidth}%
  \noalign{\global\arrayrulewidth\savedwidth}%
}

% \thickhline command for thick horizontal lines that span the table
\newcommand\thickhline{\noalign{\global\savedwidth\arrayrulewidth\global\arrayrulewidth 2pt}%
\hline
\noalign{\global\arrayrulewidth\savedwidth}}


% Remove comment for double spacing
%\usepackage{setspace}
%\doublespacing

% Text layout
\raggedright
\setlength{\parindent}{0.5cm}
\textwidth 5.25in
\textheight 8.75in

% Bold the 'Figure #' in the caption and separate it from the title/caption with a period
% Captions will be left justified
\usepackage[aboveskip=1pt,labelfont=bf,labelsep=period,justification=raggedright,singlelinecheck=off]{caption}
\renewcommand{\figurename}{Fig}

% Use the PLoS provided BiBTeX style
% \bibliographystyle{plos2015}

% Remove brackets from numbering in List of References
\makeatletter
\renewcommand{\@biblabel}[1]{\quad#1.}
\makeatother



% Header and Footer with logo
\usepackage{lastpage,fancyhdr,graphicx}
\usepackage{epstopdf}
%\pagestyle{myheadings}
\pagestyle{fancy}
\fancyhf{}
%\setlength{\headheight}{27.023pt}
%\lhead{\includegraphics[width=2.0in]{PLOS-submission.eps}}
\rfoot{\thepage/\pageref{LastPage}}
\renewcommand{\headrulewidth}{0pt}
\renewcommand{\footrule}{\hrule height 2pt \vspace{2mm}}
\fancyheadoffset[L]{2.25in}
\fancyfootoffset[L]{2.25in}
\lfoot{\today}

%% Include all macros below

\newcommand{\lorem}{{\bf LOREM}}
\newcommand{\ipsum}{{\bf IPSUM}}


% Pandoc citation processing




\usepackage{forarray}
\usepackage{xstring}
\newcommand{\getIndex}[2]{
  \ForEach{,}{\IfEq{#1}{\thislevelitem}{\number\thislevelcount\ExitForEach}{}}{#2}
}

\setcounter{secnumdepth}{0}

\newcommand{\getAff}[1]{
  \getIndex{#1}{deuc,ieb,uccs,ipe,geup,iact}
}

\providecommand{\tightlist}{%
  \setlength{\itemsep}{0pt}\setlength{\parskip}{0pt}}

\begin{document}
\vspace*{0.2in}

% Title must be 250 characters or less.
\begin{flushleft}
{\Large
\textbf\newline{Hydrological Variability in Atacama Altiplano Lakes During The Last Millennia} % Please use "sentence case" for title and headings (capitalize only the first word in a title (or heading), the first word in a subtitle (or subheading), and any proper nouns).
}
\newline
% Insert author names, affiliations and corresponding author email (do not include titles, positions, or degrees).
\\
Matías Frugone-Álvarez\textsuperscript{1,2}\textsuperscript{\getAff{deuc,ieb}},
Blas Valero-Garcés\textsuperscript{1,2}\textsuperscript{\getAff{ipe}},
Ross Williams\textsuperscript{1,2}\textsuperscript{\getAff{uccs,ciba,geup}},
David McGee\textsuperscript{1,2}\textsuperscript{\getAff{uccs,ciba,geup}},
Fernando Barreiro-Lostres\textsuperscript{1,2}\textsuperscript{\getAff{uccs,ciba,geup}},
Patricia Bernárdez\textsuperscript{1,2}\textsuperscript{\getAff{iact}},
Ricardo Prego\textsuperscript{1,2}\textsuperscript{\getAff{uccs,ciba,geup}},
Claudio Latorre\textsuperscript{1,2}\textsuperscript{\getAff{deuc}}\\
\bigskip
\textbf{\getAff{deuc}}Departamento de Ecología \& Centro UC Desierto de Atacama, Pontificia Universidad Católica de Chile, Santiago, Chile.\\
\textbf{\getAff{ieb}}Instituto de Ecología y Biodiversidady (IEB), Santiago, Chile\\
\textbf{\getAff{uccs}}Instituto Pirenaico de Ecología (IPE-CSIC), Zaragoza, Spain.\\
\textbf{\getAff{ipe}}Centro de Investigación en Biodiversidad y Ambientes Sustentables (CIBAS), Chile\\
\textbf{\getAff{geup}}Department of Geology and Environmental Science, University of Pittsburgh, Pittsburgh, USA\\
\textbf{\getAff{iact}}Instituto Andaluz de Ciencias de la Tierra (CSIC-UGR), Granada, Spain\\
\bigskip
\end{flushleft}
% Please keep the abstract below 300 words
\section*{Abstract}
Paleohydrological reconstructions from the Chilean Altiplano document abrupt moisture fluctuations during the last millennia. Although the end of the mid Holocene aridity and the onset of more humid conditions between 6--4 ka has been identified in numerous regional marine and terrestrial sites, the timing of late Holocene dry and humid phases shows large regional variability. Laguna Miscanti and Laguna Miñiques (23\(^\circ\) 43'S -- 67\(^\circ\) 46'W, 4140 m asl) are topographically closed, but connected by surface outflow from Miscanti. Sedimentological and geochemical indicators from two new cores show large facies changes, i.e.~higher carbonate and evaporite deposition during more arid periods and increased organic productivity (both algal and macrophyte) during more humid phases. As in most Andean lakes located in volcanic settings, large 14C reservoir effects occur complicating 14C dating, so the age models include 210Pb and U/Th dating. In spite of dating uncertainties, both lakes show similar patterns during the last millennium. A humid phase in Laguna Miscanti prior to ca 1200 CE is coherent with rodent middens and geomorphological features indicative of a major pluvial/recharge event at lower altitudes (Atacama Desert) during the Medieval Climate Anomaly (ca 800 - 1300 CE). The LIA (1300 -- 1850 CE) is characterized by several arid/humid cycles and the last century by a productivity increase. The hydrological changes observed during the last millennium illustrate the complex dynamics of recent climate evolution over the high altitude Andean plateau. Discrepancies between the northern and southern Altiplano records and with intermediate latitudes (Central Chile) records may reflect contrasting responses to external forcing (Westerlies versus South American Monsoon dynamics) along different climatic zones.

% Please keep the Author Summary between 150 and 200 words
% Use first person. PLOS ONE authors please skip this step.
% Author Summary not valid for PLOS ONE submissions.

\linenumbers

% Use "Eq" instead of "Equation" for equation citations.
\hypertarget{methods}{%
\section{Methods}\label{methods}}

We report how we determined our sample size, all data exclusions (if any), all manipulations, and all measures in the study.

\hypertarget{participants}{%
\subsection{Participants}\label{participants}}

\hypertarget{material}{%
\subsection{Material}\label{material}}

\hypertarget{procedure}{%
\subsection{Procedure}\label{procedure}}

\hypertarget{data-analysis}{%
\subsection{Data analysis}\label{data-analysis}}

We used R {[}\protect\hyperlink{ref-R-base}{1}{]} and the R-packages \emph{bookdown} {[}\protect\hyperlink{ref-R-bookdown}{2}{]}, \emph{citr} {[}\protect\hyperlink{ref-R-citr}{3}{]}, \emph{dplyr} {[}\protect\hyperlink{ref-R-dplyr}{4}{]}, \emph{forcats} {[}\protect\hyperlink{ref-R-forcats}{5}{]}, \emph{gamlss} {[}\protect\hyperlink{ref-R-gamlss}{6}--\protect\hyperlink{ref-R-gamlss.dist}{8}{]}, \emph{gamlss.data} {[}\protect\hyperlink{ref-R-gamlss.data}{7}{]}, \emph{gamlss.dist} {[}\protect\hyperlink{ref-R-gamlss.dist}{8}{]}, \emph{ggplot2} {[}\protect\hyperlink{ref-R-ggplot2}{9}{]}, \emph{ggpubr} {[}\protect\hyperlink{ref-R-ggpubr}{10}{]}, \emph{ggthemes} {[}\protect\hyperlink{ref-R-ggthemes}{11}{]}, \emph{gridExtra} {[}\protect\hyperlink{ref-R-gridExtra}{12}{]}, \emph{kableExtra} {[}\protect\hyperlink{ref-R-kableExtra}{13}{]}, \emph{knitr} {[}\protect\hyperlink{ref-R-knitr}{14}{]}, \emph{lattice} {[}\protect\hyperlink{ref-R-lattice}{15}{]}, \emph{MASS} {[}\protect\hyperlink{ref-R-MASS}{16}{]}, \emph{nlme} {[}\protect\hyperlink{ref-R-nlme}{17}{]}, \emph{papaja} {[}\protect\hyperlink{ref-R-papaja}{18}{]}, \emph{parallel} {[}\protect\hyperlink{ref-R-parallel}{19}{]}, \emph{plyr} {[}\protect\hyperlink{ref-R-dplyr}{4},\protect\hyperlink{ref-R-plyr}{20}{]}, \emph{png} {[}\protect\hyperlink{ref-R-png}{21}{]}, \emph{purrr} {[}\protect\hyperlink{ref-R-purrr}{22}{]}, \emph{readr} {[}\protect\hyperlink{ref-R-readr}{23}{]}, \emph{rmarkdown} {[}\protect\hyperlink{ref-R-rmarkdown_a}{24},\protect\hyperlink{ref-R-rmarkdown_b}{25}{]}, \emph{Rmisc} {[}\protect\hyperlink{ref-R-Rmisc}{26}{]}, \emph{rticles} {[}\protect\hyperlink{ref-R-rticles}{27}{]}, \emph{splines} {[}\protect\hyperlink{ref-R-splines}{28}{]}, \emph{stringr} {[}\protect\hyperlink{ref-R-stringr}{29}{]}, \emph{tibble} {[}\protect\hyperlink{ref-R-tibble}{30}{]}, \emph{tidyr} {[}\protect\hyperlink{ref-R-tidyr}{31}{]}, and \emph{tidyverse} {[}\protect\hyperlink{ref-R-tidyverse}{32}{]} for all our analyses.

\hypertarget{results}{%
\section{Results}\label{results}}

\hypertarget{discussion}{%
\section{Discussion}\label{discussion}}

\newpage

\hypertarget{references}{%
\section{References}\label{references}}

\begingroup
\setlength{\parindent}{-0.5in}
\setlength{\leftskip}{0.5in}

\hypertarget{refs}{}
\leavevmode\hypertarget{ref-R-base}{}%
1. R Core Team. R: A language and environment for statistical computing. Vienna, Austria: R Foundation for Statistical Computing; 2020. Available: \url{https://www.R-project.org/}

\leavevmode\hypertarget{ref-R-bookdown}{}%
2. Xie Y. Bookdown: Authoring books and technical documents with R markdown. Boca Raton, Florida: Chapman; Hall/CRC; 2016. Available: \url{https://github.com/rstudio/bookdown}

\leavevmode\hypertarget{ref-R-citr}{}%
3. Aust F. Citr: 'RStudio' add-in to insert markdown citations. 2019. Available: \url{https://CRAN.R-project.org/package=citr}

\leavevmode\hypertarget{ref-R-dplyr}{}%
4. Wickham H, François R, Henry L, Müller K. Dplyr: A grammar of data manipulation. 2020. Available: \url{https://CRAN.R-project.org/package=dplyr}

\leavevmode\hypertarget{ref-R-forcats}{}%
5. Wickham H. Forcats: Tools for working with categorical variables (factors). 2020. Available: \url{https://CRAN.R-project.org/package=forcats}

\leavevmode\hypertarget{ref-R-gamlss}{}%
6. Rigby RA, Stasinopoulos DM. Generalized additive models for location, scale and shape,(with discussion). Applied Statistics. 2005;54.3: 507--554.

\leavevmode\hypertarget{ref-R-gamlss.data}{}%
7. Stasinopoulos M, Rigby B, De Bastiani F. Gamlss.data: GAMLSS data. 2019. Available: \url{https://CRAN.R-project.org/package=gamlss.data}

\leavevmode\hypertarget{ref-R-gamlss.dist}{}%
8. Stasinopoulos M, Rigby R. Gamlss.dist: Distributions for generalized additive models for location scale and shape. 2020. Available: \url{https://CRAN.R-project.org/package=gamlss.dist}

\leavevmode\hypertarget{ref-R-ggplot2}{}%
9. Wickham H. Ggplot2: Elegant graphics for data analysis. Springer-Verlag New York; 2016. Available: \url{https://ggplot2.tidyverse.org}

\leavevmode\hypertarget{ref-R-ggpubr}{}%
10. Kassambara A. Ggpubr: 'Ggplot2' based publication ready plots. 2020. Available: \url{https://CRAN.R-project.org/package=ggpubr}

\leavevmode\hypertarget{ref-R-ggthemes}{}%
11. Arnold JB. Ggthemes: Extra themes, scales and geoms for 'ggplot2'. 2019. Available: \url{https://CRAN.R-project.org/package=ggthemes}

\leavevmode\hypertarget{ref-R-gridExtra}{}%
12. Auguie B. GridExtra: Miscellaneous functions for "grid" graphics. 2017. Available: \url{https://CRAN.R-project.org/package=gridExtra}

\leavevmode\hypertarget{ref-R-kableExtra}{}%
13. Zhu H. KableExtra: Construct complex table with 'kable' and pipe syntax. 2020. Available: \url{https://CRAN.R-project.org/package=kableExtra}

\leavevmode\hypertarget{ref-R-knitr}{}%
14. Xie Y. Dynamic documents with R and knitr. 2nd ed. Boca Raton, Florida: Chapman; Hall/CRC; 2015. Available: \url{https://yihui.org/knitr/}

\leavevmode\hypertarget{ref-R-lattice}{}%
15. Sarkar D. Lattice: Multivariate data visualization with r. New York: Springer; 2008. Available: \url{http://lmdvr.r-forge.r-project.org}

\leavevmode\hypertarget{ref-R-MASS}{}%
16. Venables WN, Ripley BD. Modern applied statistics with s. Fourth. New York: Springer; 2002. Available: \url{http://www.stats.ox.ac.uk/pub/MASS4/}

\leavevmode\hypertarget{ref-R-nlme}{}%
17. Pinheiro J, Bates D, DebRoy S, Sarkar D, R Core Team. nlme: Linear and nonlinear mixed effects models. 2020. Available: \url{https://CRAN.R-project.org/package=nlme}

\leavevmode\hypertarget{ref-R-papaja}{}%
18. Aust F, Barth M. papaja: Create APA manuscripts with R Markdown. 2020. Available: \url{https://github.com/crsh/papaja}

\leavevmode\hypertarget{ref-R-parallel}{}%
19. R Core Team. R: A language and environment for statistical computing. Vienna, Austria: R Foundation for Statistical Computing; 2020. Available: \url{https://www.R-project.org/}

\leavevmode\hypertarget{ref-R-plyr}{}%
20. Wickham H. The split-apply-combine strategy for data analysis. Journal of Statistical Software. 2011;40: 1--29. Available: \url{http://www.jstatsoft.org/v40/i01/}

\leavevmode\hypertarget{ref-R-png}{}%
21. Urbanek S. Png: Read and write png images. 2013. Available: \url{https://CRAN.R-project.org/package=png}

\leavevmode\hypertarget{ref-R-purrr}{}%
22. Henry L, Wickham H. Purrr: Functional programming tools. 2020. Available: \url{https://CRAN.R-project.org/package=purrr}

\leavevmode\hypertarget{ref-R-readr}{}%
23. Wickham H, Hester J. Readr: Read rectangular text data. 2020. Available: \url{https://CRAN.R-project.org/package=readr}

\leavevmode\hypertarget{ref-R-rmarkdown_a}{}%
24. Xie Y, Allaire JJ, Grolemund G. R markdown: The definitive guide. Boca Raton, Florida: Chapman; Hall/CRC; 2018. Available: \url{https://bookdown.org/yihui/rmarkdown}

\leavevmode\hypertarget{ref-R-rmarkdown_b}{}%
25. Xie Y, Dervieux C, Riederer E. R markdown cookbook. Boca Raton, Florida: Chapman; Hall/CRC; 2020. Available: \url{https://bookdown.org/yihui/rmarkdown-cookbook}

\leavevmode\hypertarget{ref-R-Rmisc}{}%
26. Hope RM. Rmisc: Rmisc: Ryan miscellaneous. 2013. Available: \url{https://CRAN.R-project.org/package=Rmisc}

\leavevmode\hypertarget{ref-R-rticles}{}%
27. Allaire J, Xie Y, R Foundation, Wickham H, Journal of Statistical Software, Vaidyanathan R, et al. Rticles: Article formats for r markdown. 2020. Available: \url{https://CRAN.R-project.org/package=rticles}

\leavevmode\hypertarget{ref-R-splines}{}%
28. R Core Team. R: A language and environment for statistical computing. Vienna, Austria: R Foundation for Statistical Computing; 2020. Available: \url{https://www.R-project.org/}

\leavevmode\hypertarget{ref-R-stringr}{}%
29. Wickham H. Stringr: Simple, consistent wrappers for common string operations. 2019. Available: \url{https://CRAN.R-project.org/package=stringr}

\leavevmode\hypertarget{ref-R-tibble}{}%
30. Müller K, Wickham H. Tibble: Simple data frames. 2020. Available: \url{https://CRAN.R-project.org/package=tibble}

\leavevmode\hypertarget{ref-R-tidyr}{}%
31. Wickham H. Tidyr: Tidy messy data. 2020. Available: \url{https://CRAN.R-project.org/package=tidyr}

\leavevmode\hypertarget{ref-R-tidyverse}{}%
32. Wickham H, Averick M, Bryan J, Chang W, McGowan LD, François R, et al. Welcome to the tidyverse. Journal of Open Source Software. 2019;4: 1686. doi:\href{https://doi.org/10.21105/joss.01686}{10.21105/joss.01686}

\endgroup

\nolinenumbers


\end{document}

